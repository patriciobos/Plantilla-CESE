% Chapter Template

\chapter{Conclusiones} % Main chapter title

\label{Chapter5} % Change X to a consecutive number; for referencing this chapter elsewhere, use \ref{ChapterX}


%----------------------------------------------------------------------------------------

%----------------------------------------------------------------------------------------
%	SECTION 1
%----------------------------------------------------------------------------------------

\section{Conclusiones generales }
El sistema desarrollado, en concordancia con el objetivo general, conforma una herramienta económica para la prestadora del servicio de energía. Esta herramienta está diseñada para otorgar mayor granularidad de información sobre el estado de operación de las redes de distribución, sin implicar cambios significativos de infraestructura. Por otra parte, el análisis de la información suministrada permite identificar eventos recurrentes y evaluar sus posibles causas para poder delinear acciones correctivas y/o preventivas para mejorar la calidad de servicio.\\
Cumplimentando todos los requerimientos planteados por el cliente, y el tiempo planteado en la planificación, se ha logrado el desarrollo exitoso del sistema en todas sus partes: \textit{hardware}, \textit{firmware}, servicios de \textit{backend}; como así también su integración con la red LoRaWAN.\\
El uso de la red LPWAN de acceso público \textit{The Things Network} seleccionada para el trabajo, ha prestado servicios durante todo el desarrollo del proyecto sin problema alguno, demostrando así su buena cobertura y calidad de servicio a nivel global. A\'{u}n habiendo cambiado la localización geográfica de Europa a Sudamérica para realizar pruebas de laboratorio, la operación del sistema no se ha visto afectada en ningún aspecto.\\
El \textit{hardware} es capaz de convertir energía de corriente alterna proveniente del transformador de intensidad en otra de corriente continua y almacenarla. Los resultados del Capítulo 4, demuestran que el uso de circuitos de \textit{energy harvesting} en conjunto con tecnologías alternativas de acumulación en constante evolución como los supercapacitores, podrían ser sustitutos factibles de las baterías litio en aplicaciones autónomas que operen en régimen 24/7 y donde el rango de temperatura de operación necesaria sea mayor.\\
Las mediciones de valor RMS de corriente realizadas en el laboratorio, simulando la señal del TI con un generador de ondas y usando una carga de prueba presentadas en \ref{ensayo_medidor_rms}, demostraron la linealidad del circuito de medición dentro del rango de medición adoptado.\\
A partir de los ensayos de consumo en modo \textit{deep sleep} y autonomía de operación presentados en el capítulo \ref{Chapter4}, queda demostrado que el patrón \textit{power save loop} ha tenido un impacto significativamente positivo en la gestión de energía del nodo.\\
El tiempo total de propagación de datos desde el nodo \textit{in situ} hacia la red LoRaWAN, recuperación por los servicios de \textit{backend} y presentación en la interfaz gr\'{a}fica de usuario, es menor a 5 segundos. Este tiempo de propagación para el reporte de un problema, es considerado excelente en contraste con la situación actual en la provincia de Misiones.\\
Un conjunto de \textit{software} con abundante documentación tal como lo es LAMPP, ha ayudado a reducir el tiempo requerido para la puesta en funcionamiento de los servicios de \textit{backend} propios del proyecto y la integración con la red LoRaWAN a través de su API REST.\\
Durante la etapa de integración entre LoRaWAN y los servicios de \textit{backend}, fue destacable la importancia de la unificación del lenguaje de programación a Python en \'{e}ste proyecto. Además de su uso para el desarrollo del \textit{firmware}, se lo utilizó para implementar mockups que emulen los datos generados por el \textit{hardware}. Mediante el uso de esta técnica se pudo garantizar un flujo de desarrollo totalmente desacoplado de la necesidad de involucrar el \textit{hardware}, pero sí con una interacción constante entre servicios \textit{web} públicos y privados.\\


%----------------------------------------------------------------------------------------
%	SECTION 2
%----------------------------------------------------------------------------------------
\section{Trabajo a futuro}

Cumplidos los requerimientos y finalizado el trabajo propuesto, se han identificado las siguientes áreas de mejoras a futuro tanto en HW como SW:

\begin{itemize}
	\item Actualizar de manera inalámbrica el \textit{firmware} (OTA - \textit{Over The Air}): nuevas versiones del \textit{firmware} del microcontrolador aportarán nuevas funcionalidades, correcciones o mejoras sobre las ya existentes en nodos desplegados. Sin embargo, desarrollar esta funcionalidad es de alta prioridad antes de que el sistema llegue a una etapa de lanzamiento de producto. De esta manera, se prescindirá de la necesidad de intervenir físicamente cada nodo para actualizarlo.\\
	\item Modularizar el PCB para realizar mediciones de 3 fases: dado que los sistemas de distribución son trifásicos, el HW deberá también permitir realizar mediciones de corrientes sobre las 3 fases del sistema. Para lograr esto se debería proponer una modularización de la etapa de medición de valor RMS de corriente.\\
	\item Integrar servicios de mensajería instantánea: si bien la GUI permite de manera rápida identificar sobre un mapa el punto geográfico donde la red presenta un problema o su estado actual de operación, contar con una aplicación similar para dispositivos móviles será de utilidad para el personal encargado de cumplir horarios de guardia.\\
	
\end{itemize}

